\documentclass[a4paper]{scrartcl}

\usepackage[utf8]{inputenc}
\usepackage[english]{babel}
\usepackage{titlesec}
\usepackage{hyperref}

\titleformat{\section}{\huge\scshape}{\thesection}{1em}{}

\newcommand{\HRule}{\rule{\linewidth}{0.5mm}}

\begin{document}

\begin{titlepage}

\begin{center}

\Large{\textsc{Report}}\\[0.4cm]

\HRule \\[0.5cm]

\Huge{\bfseries{Segmenting an Aneurysm of the Abdominal Aorta}}\\[0.5cm]

\Large{\textsc{April 2014}}\\[0.4cm]

\HRule \\[0.5cm]

\vfill

\center{\textsc{\Large{Jan Henric Klau}}}\\[0.5cm]  
\center{\textsc{\Large{Tim Shirley}}}\\[0.3cm]                                  

\vfill

\end{center}

\vfill
\begin{center}
    \small{\textsc{Faculy IV - Universität Siegen}}\\[0.4cm]
    \small{\textsc{Research Group for Pattern Recognition}}\\[0.4cm]                
\end{center}

\end{titlepage}


\section*{Task}

The task was to segment the Aorta wall around an aneurysm of the abdominal
aorta using an algorithm implemented in Matlab. 

This document will briefly describe our initial approaches, give a detailed
description of our final implementation and suggest improvements to our
algorithms.

\section*{Approach}

The first idea was to use a simple threshold to segment the bright parts of the
contrast-enhanced aorta. While the inner part of the aorta shows high contrast,
the outer part of the aorta does not, making this approach valuable for finding
contrast-enhanced arteries but not the aneurysm itsself.

Due to the high contrast of the aortic blood finding the egde of the aorta
using classic edge-detection algorithms should have been relatively easy. 
However the wall of the aorta prooved to be too similar to the surrounding 
tissue and changing parameters of the algorithm only resulted in either too
many or to few detected edges.

Due to the mostly very regular and circular shape of the aorta we quickly 
considered using a knowledge-based approach to edge-detection. The idea was to
use a Hough Transform to detect circular shapes within a certain range of
diameters. Given a slice with high contrast along the aorta wall, a manually
defined region of interest containing the aorta and the correct range of radii
this approach did indeed find the aorta wall quite robustly. Extrapolating from
strong matches from one sclice to the next would probably allow for reasonable
results. However, the necessety for manual interaction, as well as the quite
haphazard assumption of a circular aorta led us to disregard this approach.

The most promising approach, and the one we chose for our final solution, was
to (manually) define a region within the aorta and use a region growing
algorithm to match the actual border of the aneurysm. Due to the very
inconsistent gradient of the edge aorta wall we quickly decided to use the a
level set evolution algorithm, even though the standard implementations didn't
produce desireable results. 

\section*{Solution}

After some research we came upon the Distance Regularized Level Set Evolution
(DRLSE) algorithm by Chunming Li~\cite{Li_TIP08}.

At first a region of interest has to be selected. Within this an inital region is chosen and being handed to the DRLSE method.
The level set method grows until the boundary matches the next edge to find.

Therefore two integers are set, the number of inner iterations and the number of outer iterations.
The number of outer iterations defines how often the level set method is used, while the number of inner iterations defines how many steps each level set method computes.

In our implementation the level set method is used twice.
First a threshold is used to get rid of the unwanted structures.
In the next step the result of the threshold is being eroded to remove remaining small structures.
After this only the high contrast area of the aorta and some parts of the spine remain.
In the next step the region of interest has to be selected from the original and the boundary of the eroded aorta is shown within.

This boundary is then used to initalize the fist level set method.
As a result of this the full high contrasted, inner part of the aneurysm is segmented.
Because a next level set method would converge to the same boundary the contour is dilated. This is done by an erosion method because the result of the level set method comes in an inverted form.
For the second level set method this new dilated boundary is used to initialize.
After the region growing the resulting boundary now matches the aorta wall.

\section*{Suggestions for further work}

\section*{Appendix: Structure of implementation}

The Matlab implementation of our segmentation algorithm can be found on github
at \url{https://github.com/procrastonos/aortic-aneurysm}.

While the GUI is implemented within the \texttt{aneurysm.m} file, all functions
implementations lay in their own 'm' files within the \emph{private}-folder.
The titles of the files should be self-explanatory. The functions of interest
are \texttt{process.m}, \texttt{levelset.m} and \texttt{circle.m}, which in
turn contain the final algorithm, the DRLSE and the circle matching via
Hough-Transform.

\bibliographystyle{alpha}
\bibliography{ref}

\end{document}
